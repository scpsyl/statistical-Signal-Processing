\documentclass[12pt]{ctexart}

% 包含必要的包
\usepackage[utf8]{inputenc}
\usepackage{amsmath, amssymb}  % 数学符号包
\usepackage{graphicx}  % 插入图片的包
\usepackage{hyperref}  % 生成超链接
\usepackage{fancyhdr}  % 页眉页脚设置
\usepackage{geometry}  % 页面设置
\usepackage{titlesec}  % 用于自定义 section 的格式
\geometry{a4paper, margin=1in}

\titleformat{\section}[hang]{\normalfont\Large\bfseries}{\thesection}{1em}{}


% Header and footer
\setlength{\headheight}{14.49998pt}
\addtolength{\topmargin}{-2.49998pt}

\pagestyle{fancy}
\fancyhf{}
\fancyhead[L]{《统计信号处理》实验报告}
% \fancyhead[M]{清华大学}
\fancyhead[R]{刘昱杉 2024214103}
\fancyfoot[C]{\thepage}



\begin{document}

\begin{titlepage}
    \begin{center}
        % Insert logo
        \includegraphics[width=5cm]{tsinghua_logo.png}\\[4cm]  % 插入图标并设置下方间距
        {\Huge 实验一:信号检测与分类} \\[4cm]
        {\large 刘昱杉  \ \  2024214103}\\[6cm]
        {\normalsize \today}\\[1cm]
    \end{center}
\end{titlepage}

\section*{一、单电平信号检测}
根据项目周期的长短,不同项目的存在时间不同,主要包括五个重要阶段:形成、磨合、规范、执行、解体。

\begin{figure}[h]
    \centering
    % \includegraphics[width=1.0 \textwidth]{steps.png}
    \label{fig:sample}
\end{figure}
\begin{enumerate}
    \item 形成阶段:需要对成员充分定位,并指定目标、分配任务。负责人需要进行项目组织的指导与构建,包括进度计划、组织结构、各个成员充当角色等任务。
    \item 磨合阶段:成员执行所分配的任务。负责人需要明确每个成员的具体职责,并尽可能地减少成员之间的冲突,构造和谐的工作环境。
    \item 规范阶段:组织目标更加明确,成员之间相互信任相互帮助,项目组织的运作更加规范。
    \item 执行阶段:项目运作步入正轨,各个成员基于完成项目目标,按照项目计划进行工作。
    \item 解体阶段:项目完成,项目组织解散,成员回归原单位。
\end{enumerate}

\section*{5-7 如何理解项目的融资成本与资本结构?}

\subsection*{融资成本}
融资成本是指企业为筹集资金而支付的成本,包括债务成本和权益成本。债务成本是指企业通过借款筹集资金所支付的利息,权益成本是指企业通过发行股票筹集资金所支付的股利。

\subsection*{资本结构}
资本结构是指企业通过债务和权益筹集资金的比例。资本结构的优劣直接影响企业的融资成本和企业的价值。资本结构的优劣取决于企业的经营风险、融资成本、税收政策等因素。


\end{document}